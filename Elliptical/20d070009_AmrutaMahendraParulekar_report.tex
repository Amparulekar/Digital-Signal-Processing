\documentclass[12pt]{article}
\usepackage{graphicx}
\usepackage{float}
\usepackage{hyperref}
\usepackage{tfrupee}
\usepackage{ragged2e}
\usepackage{multicol}
\usepackage{multirow}
\usepackage{xurl}
\usepackage{geometry}
\usepackage{enumitem}
\documentclass{minimal}
\usepackage{mathtools}
\DeclarePairedDelimiter\ceil{\lceil}{\rceil}
\DeclarePairedDelimiter\floor{\lfloor}{\rfloor}
 \geometry{
 a4paper,
 total={170mm,240mm},
 left=20mm,
 top=20mm,
 }
% begin the document.
\begin{document}

% make a title page.[this creates title page]
\begin{center}\textbf{
\LARGE{EE338 : Digital Signal Processing}\\\vspace{1em}
\vspace{1em} \\ 
\LARGE{Filter Design Assignment IV}\\
\vspace{1em}
\large{Marh 24, 2023}\\
\vfill
\begin{figure}[H]
    \centering
    \includegraphics[scale = 0.2]{logo.png}
\end{figure}
\vspace{1em}
\large{Indian Institute of Technology, Bombay}\\
\vfill
\LARGE{Amruta Mahendra Parulekar} \\
\LARGE{Roll no. 20d070009} \vspace{2em}
\\\large{Reviewed by Harshvardhan, 20d070035}

}
\end{center}

\newpage
\tableofcontents
\newpage
\section{Student Details} %[This segment creates Section as seen in document]
\begin{itemize}[nolistsep]
    \item Name = Amruta Mahendra Parulekar
    \item Roll no. = 20d070009
    \item Filter number m = 26
    \item Group number = 3
    \item Review member = Harshvardhan, 20d070035 (Has reviewed my report)
\end{itemize}
\section{Bandpass filter Design}
\subsection{Un-normalized discrete time filter specifications}

  
    The filter to be designed is a Band-pass filter where:
    \begin{equation}
        q(m) = \floor{m/10} = \floor{2.6} = 2
    \end{equation}
    \begin{equation}
        r(m) = 26 - 10*q(m) = 26 - 10*2 = 6
    \end{equation}
    \begin{equation}
        BL(m) = 10 + 5*q(m) + 13*r(m) = 10 + 5*2 + 13*6 = 98
    \end{equation}
    \begin{equation}
        BH(m) = BL(m) + 75 = 98 + 75 = 173
    \end{equation}
  

    \begin{enumerate}
    \item \textbf{The passband will be equiripple and the stopband will be equiripple}
    \item \textbf{The passband will be from 98 kHz to 173 kHz}
    \item \textbf{The transition band will be 5 kHz on either side of the passband}
    \item \textbf{The stopband will be from 0 - 93 kHz and 178 - 300kHz} ( sampling rate 600 kHz)
    \item \textbf{The passband and stopband tolerances are 0.15 in magnitude}

\end{enumerate}
\subsection{Normalized discrete time filter specifications}
Sampling rate is 600 kHz, which corresponds to 2$\pi$ on the normalized frequency axis.
\\So on normalizing the frequency axis, each frequency $\Omega_1$ below 300 kHz gets mapped using the function:
\begin{equation}
    \omega=\frac{\Omega_1 * 2 \pi}{(Sampling Rate)}
\end{equation}
\begin{enumerate}
    \item \textbf{The passband will be equiripple and the stopband will be equiripple}
    \item \textbf{The passband will be from 0.3267 $\pi$ to 0.5767 $\pi$}
    \item \textbf{The transition band will be 0.017 $\pi$ on either side of the passband}
    \item \textbf{The stopband will be from 0 $\pi$ - 0.31 $\pi$ and 0.593 $\pi$- $\pi$}
    \item \textbf{The passband and stopband tolerances are 0.15 in magnitude}

\end{enumerate}
\newpage
\subsection{Analog filter specifications }
The discrete time filter specifications can be converted to corresponding analog filter specifications by using a bilinear transform, which is given as:
\begin{equation}
    \Omega = tan(\omega/2)
\end{equation}
\begin{enumerate}
    \item \textbf{The passband will be equiripple and the stopband will be equiripple}
    \item \textbf{The passband will be from 0.5635($\Omega_{p1}$) to 1.2754($\Omega_{p2}$)}
    \item \textbf{The transition band will be from 0.5295($\Omega_{s1}$) - 0.5635($\Omega_{p1}$) and \\from 1.2754($\Omega_{p2}$)-1.345($\Omega_{s2}$)}
    \item \textbf{The stopband will be from 0 - 0.5295($\Omega_{s1}$) and 1.345($\Omega_{s2}$) - infinity}
    \item \textbf{The passband and stopband tolerances are 0.15 in magnitude}

\end{enumerate}
\subsection{The frequency transformation }
We use the bandpass transformation to convert the band pass filter to a lower filter:
\begin{equation}
    \Omega_L=\frac{\Omega^2-\Omega_{0}^2}{B\Omega}
\end{equation}
The two parameters B and $\Omega_{0}$ are obtained by the relations:
\begin{equation}
    \Omega_{0}=\sqrt{\Omega_{p1}*\Omega_{p2}}= \sqrt{0.5635*1.2753}=0.8478
    \end{equation}
    \begin{equation}   
    B=\Omega_{p2}-\Omega_{p1}=0.7118
 \end{equation}
 \begin{center}
\def\arraystretch{1.1}
\bgroup
\begin{tabular}{|c|c|}
\hline
\textbf{$\Omega$ }
& \textbf{$\Omega_L$ }\\ 
\hline \hline
$0^+$  &  -infinity  \\ 
\hline 
0.5295 ($\Omega_{s1}$)   &   -1.163 ($\Omega_{Ls1}$) \\ 
\hline
 0.5635 ($\Omega_{p1}$) &    -1 ($\Omega_{Lp1}$)\\ 
\hline
0.8478  ($\Omega_{0}$) & 0  \\ 
\hline
1.2753 ($\Omega_{p2}$)   &  1 ($\Omega_{Lp2}$) \\ 
\hline
 1.345 ($\Omega_{s2}$) &   1.139 ($\Omega_{Ls2}$) \\ 
 \hline
 infinity & infinity \\
 \hline
\end{tabular}
\egroup
\end{center}
\subsection{Frequency transformed lowpass analog filter specifications}
\begin{enumerate}
    \item \textbf{The passband will be equiripple and the stopband will be equiripple}
    \item \textbf{The passband edge will be at 1($\Omega_{Lp}$)}
    \item \textbf{The stopband edge will be min(-$\Omega_{Ls1}$,$\Omega_{Ls2}$) which is 1.139($\Omega_{Ls}$)}
    \item \textbf{The passband and stopband tolerances are 0.15 in magnitude}

\end{enumerate}

\newpage
\subsection{The analog lowpass filter transfer function}
\subsubsection{Calculation of filter order}
We require a Elliptical Bandpass filter, which means that both the passband and stopband are equiripple.
\\First, the passband edge is normalized to 1 and the maximum passband response is set at one. Set the
stopband edge and the maximum ripple in the passband and stopband. The filter order will be calculated from
these constraints.
\\Since the tolerance ($\delta$) in both passband and stopband is 0.15, we define parameters as:
\begin{equation}
    \epsilon = \sqrt{\frac{2*\delta-\delta^2}{1-2*\delta+\delta^2}} = 0.6197
\end{equation}
\begin{equation}
    k_1=\frac{\epsilon}{\sqrt{\frac{1}{\delta^2}-1}}=0.09402
\end{equation}
\begin{equation}
    k'_1=\sqrt{1-k_1^2}=0.9955
\end{equation}
\begin{equation}
    k=\frac{\Omega_{Lp}}{\Omega_{Ls}}=0.8779
\end{equation}
\begin{equation}
    k'=\sqrt{1-k^2}=0.4788
\end{equation}
For correct calculations, check to see that k and $k'_1$ are not too close to 1. For accuracy, they should be less
than 1 -$ 10^{-9}$
\\The inequality for the order N of the Elliptical filter is:
\begin{equation}
    N_{min}=\ceil{\frac{K(k)*K(k'_1)}{K(k')*K(k_1)}} 
\end{equation}
Where 
\begin{equation}
    K(k)=\int^{\pi/2}_0 \frac{d\theta}{\sqrt{1-k^2sin^2\theta}}
\end{equation}
On calculating and approximating the integral,
\begin{equation}
    K(k'_1)=3.7477;  K(k_1)=1.5742; K(k)=2.1953; K(k')=1.6746
\end{equation}
Substituting in (15),
\begin{equation}
    N_{min}=\ceil{3.1209}=4
\end{equation}
\subsubsection{Finding poles and zeroes of the transfer function}
\href{https://community.ptc.com/sejnu66972/attachments/sejnu66972/PTCMathcad/176201/1/13.2_Analog_Elliptic_Filter_Design.pdf}{Reference}
The elliptical functions:
\begin{equation}
    U(\phi,k)=1i*\int^{Im(\phi)}_0 \frac{d\theta}{\sqrt{1-k^2sin^2(1j*\theta)}}+\int^{Re(\phi)}_0 \frac{d\theta}{\sqrt{1-k^2sin^2(1j*Im(\phi)+\theta)}}
\end{equation}
\begin{equation}
    \phi=1i ; sn(u,k)=sine(root(U(\phi,k)-u,\phi))
\end{equation}
\begin{equation}
 v=\frac{K(k)}{N*N(k_1)}*U(atan(\frac{1}{\epsilon},k'_1))
\end{equation}
Since N is even,
\begin{equation}
m=0,1,....N-1; c_m=1+2*floor(m/2)
\end{equation}
Roots:
\begin{equation}
Sr_m=\frac{(-1)^m*1j}{k*sn(c_m*K(k)/N,k)}
\end{equation}
We get the roots as: \textbf{\\-1.11022302462516e-16 + 0.939929938460878i
\\-1.11022302462516e-16 - 0.939929938460878i
\\0.00000000000000 + 0.565295219831206i
\\0.00000000000000 - 0.565295219831206i}
\\Since N is even,
\begin{equation}
l=0,1,....N-1; d_l=1-mod(N,2)+2*floor((l+mod(N,2))/2)
\end{equation}
Poles:
\begin{equation}
Sp_l=conj(1j*sn(\frac{d_l*K(k)}{N}+1j*v,k),l)
\end{equation}
\\We get the poles as: \textbf{\\-0.565693012906616 + 1.13130568970048i
\\-0.565693012906616 - 1.13130568970048i
\\-0.0309599575375007 + 0.999508300849399i
\\-0.0309599575375007 - 0.999508300849399i}

\subsubsection{Polynomial Expansion}
We can write the transfer function as (N even):
\begin{equation}
    H_{analog,LPF}(s_L) = 0.85*\frac{\prod_{l=1}^{4}(Sp_l)}{\prod_{m=1}^{4}(Sr_m)}*\frac{\prod_{m=1}^{4}(s_L-Sr_m)}{\prod_{l=1}^{4}(s_L-Sp_l)} = 4.784*\frac{\prod_{m=1}^{4}(s_L-Sr_m)}{\prod_{l=1}^{4}(s_L-Sp_l)}
\end{equation}
The numerator has been scaled to control DC gain

\subsection{The analog bandpass filter transfer function}
The transformation equation:
\begin{equation}
    s_L= \frac{s^2+\Omega_{0}^2}{Bs} = \frac{s^2+0.8478^2}{0.7118*s}
\end{equation}
Thus, the transfer function:
\begin{equation}
    H_{analog,BPF}(s)=4.784*\frac{\prod_{i=0}^{7}(s^2-Sr_m*Bs+\Omega_{0}^2)}{\prod_{i=0}^{7}(s^2-Sp_l*Bs+\Omega_{0}^2)} 
\end{equation}
Factorizing the denominator and numerator to get new roots so that the each can be expressed as a product of 8 monomials, we get the new poles and roots as:
\begin{equation}
    pol_i=\frac{Sp_lB\pm \sqrt{(Sp_lB)^2-4\Omega_{0}^4}}{2}
\end{equation}
\begin{equation}
    roo_i=\frac{Sr_mB\pm \sqrt{(Sr_mB)^2-4\Omega_{0}^4}}{2}
\end{equation}
Thus the new analog filter transfer function where $pol_i$ are poles and $rot_i$ are the roots of the analog bandpass filter is:
\begin{equation}
    H_{analog,BPF}(s)=4.784*\frac{\prod_{i=0}^{7}(s-rot_i)}{\prod_{i=0}^{7}(s-pol_i)} 
\end{equation}
\\We get the new poles as: \textbf{\\-0.0152825778461489 + 1.27508577763069i
\\-0.0152825778461489 - 1.27508577763069i
\\-0.161990969014231 + 1.12777817048717i
\\-0.161990969014231 - 1.12777817048717i
\\-0.0896935518516070 + 0.624444871385489i
\\-0.0896935518516070 - 0.624444871385489i
\\-0.00675526288771787 + 0.563618240260224i
\\-0.00675526288771787 - 0.563618240260224i}
\\We get the new roots as: \textbf{\\-1.11022302462516e-16 + 1.68558396640932i
\\-1.11022302462516e-16 - 1.68558396640932i
\\2.77555756156289e-16 + 1.30715860326007i
\\2.77555756156289e-16 - 1.30715860326007i
\\6.93889390390723e-18 + 0.549868117156857i
\\6.93889390390723e-18 - 0.549868117156857i
\\-4.16333634234434e-17 + 0.426418887651817i
\\-4.16333634234434e-17 - 0.426418887651817i}
    
\subsection{The discrete time filter transfer function}
The bilinear transform from analog to discrete domain is:
\begin{equation}
    \frac{1-z^{-1}}{1+z^{-1}}
\end{equation}
On substituting this s in equation for the analog Bandpass filter, we get:
\begin{equation}
    H_{discrete,BPF}(s)=4.784*\frac{\prod_{i=0}^{7}((1-rot_i)-(1+rot_i)z^{-1})}{\prod_{i=0}^{7}((1-pol_i)-(1+pol_i)z^{-1})} 
\end{equation}
Thus we get new the poles at:
\begin{equation}
    p_i=\frac{1+pol_i}{1-pol_i}
\end{equation}
Thus we get new the roots at:
\begin{equation}
    r_i=\frac{1+rot_i}{1-rot_i}
\end{equation}
\newpage
 \\We get the new poles as: \textbf{\\0.512528345853624 + 0.846768421342538i
\\0.512528345853624 - 0.846768421342538i
\\0.381664527177259 + 0.791757762083732i
\\0.381664527177259 - 0.791757762083732i
\\-0.235664869673740 + 0.959922759720789i
\\-0.235664869673740 - 0.959922759720789i
\\-0.113696626326866 + 0.860207715818710i
\\-0.113696626326866 - 0.860207715818710i}
\\We get the new roots as: \textbf{\\0.692286376631899 + 0.721622874311699i
\\0.692286376631899 - 0.721622874311699i
\\0.535679659170337 + 0.844421282744076i
\\0.535679659170337 - 0.844421282744076i
\\-0.479328469116527 + 0.877635584222977i
\\-0.479328469116527 - 0.877635584222977i
\\-0.261628505804086 + 0.965168651040181i
\\-0.261628505804086 - 0.965168651040181i}

\\We have the values of all coefficients for equation (33) and thus we can substitute and find the values of the coefficients of the discrete bandpass filter transfer function.
\begin{center}
\includegraphics[scale=0.4]{bb.jpg}
\end{center}

\subsection{Code}
\begin{verbatim}
close all;clearvars;clc;

%Specifications
fp1 = 98;fs1 = 93;fs2 = 178;fp2 = 173;f_samp = 600;
B1=0.7118;W0=0.8478;
Ws=1.139;Wp=1;
Gp = 0.85;Gs = 0.15;


%Calculating N
ep = sqrt(1/Gp^2 - 1); es = sqrt(1/Gs^2 - 1);
k = Wp/Ws ;k1 = ep/es;
[K,Kp] = ellipk(k);[K1,K1p] = ellipk(k1);
Nexact = (K1p/K1)/(Kp/K);N = ceil(Nexact); 

%Elliptical approximation
Ap=1.4116;As=16.4782;
[z,p,H0,B,A] = ellipap2(N,Ap,As);
[y1,~] = size(B);
num = B(1,:);den = A(1,:);
if y1>1
    for i=2:y1
        num=conv(num,B(i,:));
        den=conv(den,A(i,:));
    end
end
syms s z;

%Analog LPF
analog_lpf(s) = poly2sym(num,s)/poly2sym(den,s);  
[ns, ds] = numden(analog_lpf(s));                                    
ns = sym2poly(expand(ns));ds = sym2poly(expand(ds));                     
kd = ds(1);kn = ns(1);    
ds = ds/k;ns = ns/k;
fvtool(ns,ds,'Analysis','freq');                       
[zs,ps,]=tf2zp(ns,ds);

%Analog BPF
analog_bpf(s) = analog_lpf((B1*s)/(s*s + W0*W0));   
[nsb, dsb] = numden(analog_bpf(s));                                      
nsb = sym2poly(expand(nsb));dsb = sym2poly(expand(dsb));                 
kd = dsb(1);kn = nsb(1);    
dsb = dsb/k;nsb = nsb/k;
fvtool(nsb,dsb,'Analysis','freq');                       
[zsb,psb,]=tf2zp(nsb,dsb);
k=kn/kd;

%Discrete BPF
discrete_bpf(z) = analog_bpf((z-1)/(z+1));              
[nz, dz] = numden(discrete_bpf(z));                                        
nz = sym2poly(expand(nz));dz = sym2poly(expand(dz));                    
k = dz(1);    
dz = dz/k;nz = nz/k;
fvtool(nz,dz,'Analysis','freq');                       
[z,p,]=tf2zp(nz,dz);

%Pole zero plot
figure;
plot(real(p),imag(p),'rX');
hold on
plot(real(z),imag(z),'bO');
title("Poles and roots of the transfer function")
xlabel("Real")
ylabel("Imaginary")
axis equal
grid on
t = linspace(0,2*pi,1000);
hold on
plot(cos(t),sin(t),'b-') 

%Magnitude and phase plot
[H,f] = freqz(nz,dz,1024*1024, f_samp);
figure;
plot(f,abs(H),'LineWidth',1);
hold on;
title("Magnitude Response vs unnormalized frquency")
xlabel("1000 Hz")
ylabel("Response")
xline(fs1);xline(fp1);xline(fp2);xline(fs2);
yline(0.85);yline(0.15);
grid
\end{verbatim}
\subsection{Comparison of results with Butterworth filter}
\subsubsection{Magnitude vs unnormalized frequency response }
\begin{center}
\includegraphics[scale=0.4]{ba.jpg}
\end{center}
\newpage
\textbf{Corresponding Butterworth graph}
\begin{center}
\includegraphics[scale=1]{p4 (1).jpg}
\end{center}
\subsubsection{Magnitude and phase response vs normalized frequency}
\begin{center}
\includegraphics[scale=0.4]{bc.jpg}
\end{center}
\textbf{Corresponding Butterworth graph}
\begin{center}
\includegraphics[scale=0.7]{p1 (1).jpg}
\includegraphics[scale=0.7]{p2 (1).jpg}
\end{center}
\subsubsection{Comparisons}
\begin{itemize}
    \item In both realizations, the specifications are satisfied and we can observe fs1,fs2,fp1 \& fp2.
    \item The Butterworth filter has a monotonic passband and a monotonic stopband while the elliptical filter has equiripple passband and stopband.
    \item The Butterworth filter had an order of 38 in both the numerator and denominator, thus requiring around 76 delay stages. For the elliptical filter, we have a power of 8 in both numerator and denominator, thus requiring only around 16 delays. Thus, the elliptical filter is more resource efficient than the butterworth filter.(lower order)
\end{itemize}

\section{Bandstop filter Design}
\subsection{Un-normalized discrete time filter specifications}

  
    The filter to be designed is a Band-stop filter where:
    \begin{equation}
        q(m) = \floor{m/10} = \floor{2.6} = 2
    \end{equation}
    \begin{equation}
        r(m) = 26 - 10*q(m) = 26 - 10*2 = 6
    \end{equation}
    \begin{equation}
        BL(m) = 20 + 3*q(m) + 11*r(m) = 20 + 3*2 + 11*6 = 92
    \end{equation}
    \begin{equation}
        BH(m) = BL(m) + 40 = 92 + 40 = 132
    \end{equation}
  

    \begin{enumerate}
    \item \textbf{The passband will be equiripple and the stopband will be equiripple}
    \item \textbf{The stopband will be from 92 kHz to 132 kHz}
    \item \textbf{The transition band will be 5 kHz on either side of the stopband}
    \item \textbf{The passband is from 0 - 87 kHz and 137 - 212.5kHz} ( sampling rate 425 kHz)
    \item \textbf{The passband and stopband tolerances are 0.15 in magnitude}

\end{enumerate}
\subsection{Normalized discrete time filter specifications}
Sampling rate is 425 kHz, which corresponds to 2$\pi$ on the normalized frequency axis.
\\So on normalizing the frequency axis, each frequency $\Omega_1$ below 212.5 kHz gets mapped using the function:
\begin{equation}
    \omega=\frac{\Omega_1 * 2 \pi}{(Sampling Rate)}
\end{equation}
\begin{enumerate}
    \item \textbf{The passband will be equiripple and the stopband will be equiripple}
    \item \textbf{The stopband will be from 0.4329 $\pi$ to 0.6212 $\pi$}
    \item \textbf{The transition band will be 0.0235 $\pi$ on either side of the stopband}
    \item \textbf{The passband will be from 0 - 0.4094 $\pi$ and 0.6447 $\pi$ - $\pi$} 
    \item \textbf{The passband and stopband tolerances are 0.15 in magnitude}

\end{enumerate}

\subsection{Analog filter specifications }
The discrete time filter specifications can be converted to corresponding analog filter specifications by using a bilinear transform, which is given as:
\begin{equation}
    \Omega = tan(\omega/2)
\end{equation}
\begin{enumerate}
    \item \textbf{The passband will be equiripple and the stopband will be equiripple}
    \item \textbf{The stopband will be from 0.8086 ($\Omega_{s1}$) to 1.4774($\Omega_{s2}$)}
    \item \textbf{The transition band will be from 0.7493 ($\Omega_{p1}$) - 0.8086 ($\Omega_{s1}$) and \\1.4774($\Omega_{s2}$) - 1.6018 ($\Omega_{p2}$)}
    \item \textbf{The passband will be from 0 - 0.7493 ($\Omega_{p1}$)  and 1.6018 ($\Omega_{p2}$) - infinity} 
    \item \textbf{The passband and stopband tolerances are 0.15 in magnitude}
\end{enumerate}
\subsection{The frequency transformation }
We use the bandstop transformation to convert the band pass filter to a lower filter:
\begin{equation}
    \Omega_L=\frac{B\Omega}{\Omega_{0}^2-\Omega^2}
\end{equation}
The two parameters B and $\Omega_{0}$ are obtained by the relations:
\begin{equation}
    \Omega_{0}=\sqrt{\Omega_{p1}*\Omega_{p2}}= \sqrt{0.7493*1.6108}=1.099
    \end{equation}
    \begin{equation}   
    B=\Omega_{p2}-\Omega_{p1}=0.8615
 \end{equation}
 \begin{center}
\def\arraystretch{1.1}
\bgroup
\begin{tabular}{|c|c|}
\hline
\textbf{$\Omega$ }
& \textbf{$\Omega_L$ }\\ 
\hline \hline
$0^+$  &  $0^+$  \\ 
\hline 
0.7493 ($\Omega_{p1}$)   &   +1 ($\Omega_{Lp1}$) \\ 
\hline
0.8086 ($\Omega_{s1}$)  &   1.2578  ($\Omega_{Ls1}$)\\ 
\hline
1.099  ($\Omega_{0-}$) & +infinity  \\ 
\hline
1.099  ($\Omega_{0+}$) & -infinity  \\ 
\hline
1.4774 ($\Omega_{s2}$)   &  -1.3029 ($\Omega_{Ls2}$) \\ 
\hline
1.6108 ($\Omega_{p2}$)  &   -1 ($\Omega_{Lp2}$) \\ 
 \hline
 infinity & $0^-$ \\
 \hline
\end{tabular}
\egroup
\end{center}
\subsection{Frequency transformed lowpass analog filter specifications}
\begin{enumerate}
    \item \textbf{The passband will be equiripple and the stopband will be equiripple}
    \item \textbf{The passband edge will be at 1($\Omega_{Lp}$)}
    \item \textbf{The stopband edge will be min($\Omega_{Ls1}$,$-\Omega_{Ls2}$) which is 1.2578($\Omega_{Ls}$)}
    \item \textbf{The passband and stopband tolerances are 0.15 in magnitude}

\end{enumerate}

\subsection{The analog lowpass filter transfer function}
We require a Elliptical Bandstop filter, which means that both the passband and stopband are equiripple.
\\First, the passband edge is normalized to 1 and the maximum passband response is set at one. Set the
stopband edge and the maximum ripple in the passband and stopband. The filter order will be calculated from
these constraints.
\\Since the tolerance ($\delta$) in both passband and stopband is 0.15, we define parameters as:
\begin{equation}
    \epsilon = \sqrt{\frac{2*\delta-\delta^2}{1-2*\delta+\delta^2}} = 0.6197
\end{equation}
\begin{equation}
    k_1=\frac{\epsilon}{\sqrt{\frac{1}{\delta^2}-1}}=0.09402
\end{equation}
\begin{equation}
    k'_1=\sqrt{1-k_1^2}=0.9955
\end{equation}
\begin{equation}
    k=\frac{\Omega_{Lp}}{\Omega_{Ls}}=0.7950
\end{equation}
\begin{equation}
    k'=\sqrt{1-k^2}=0.6065
\end{equation}
For correct calculations, check to see that k and $k'_1$ are not too close to 1. For accuracy, they should be less
than 1 -$ 10^{-9}$
\\The inequality for the order N of the Elliptical filter is:
\begin{equation}
    N_{min}=\ceil{\frac{K(k)*K(k'_1)}{K(k')*K(k_1)}} 
\end{equation}
Where 
\begin{equation}
    K(k)=\int^{\pi/2}_0 \frac{d\theta}{\sqrt{1-k^2sin^2\theta}}
\end{equation}
On calculating and approximating the integral,
\begin{equation}
    K(k'_1)=3.7477;  K(k_1)=1.5742; K(k)=1.9857; K(k')=1.7558
\end{equation}
Substituting in (15),
\begin{equation}
    N_{min}=\ceil{2.6924}=3
\end{equation}

\subsubsection{Finding poles and zeroes of the transfer function}
\href{https://community.ptc.com/sejnu66972/attachments/sejnu66972/PTCMathcad/176201/1/13.2_Analog_Elliptic_Filter_Design.pdf}{Reference}
The elliptical functions:
\begin{equation}
    U(\phi,k)=1i*\int^{Im(\phi)}_0 \frac{d\theta}{\sqrt{1-k^2sin^2(1j*\theta)}}+\int^{Re(\phi)}_0 \frac{d\theta}{\sqrt{1-k^2sin^2(1j*Im(\phi)+\theta)}}
\end{equation}
\begin{equation}
    \phi=1i ; sn(u,k)=sine(root(U(\phi,k)-u,\phi))
\end{equation}
\begin{equation}
 v=\frac{K(k)}{N*N(k_1)}*U(atan(\frac{1}{\epsilon},k'_1))
\end{equation}
Since N is odd,
\begin{equation}
m=0,1,....N-2; c_m=2+2*floor(m/2)
\end{equation}
Roots:
\begin{equation}
Sr_m=\frac{(-1)^m*1j}{k*sn(c_m*K(k)/N,k)}
\end{equation}
We get the roots as: \textbf{\\0.00000000000000 + 0.00000000000000i
\\-0.00000000000000 + 0.793401938900610i
\\0.00000000000000 - 0.793401938900610i}
\\Since N is odd,
\begin{equation}
l=0,1,....N-1; d_l=1-mod(N,2)+2*floor((l+mod(N,2))/2)
\end{equation}
Poles:
\begin{equation}
Sp_l=conj(1j*sn(\frac{d_l*K(k)}{N}+1j*v,k),l)
\end{equation}
\\We get the poles as: \textbf{\\-1.60473946109342 + 0.00000000000000i
\\-0.115266491897389 + 0.993048663800526i
\\-0.115266491897389 - 0.993048663800526i}

\subsubsection{Polynomial Expansion}
We can write the transfer function as (N odd):
\begin{equation}
    H_{analog,LPF}(s_L) = 1*\frac{\prod_{l=1}^{3}(Sp_l)}{\prod_{m=1}^{3}(Sr_m)}*\frac{\prod_{m=1}^{3}(s_L-Sr_m)}{\prod_{l=1}^{4}(s_L-Sp_l)} = 2.5492*\frac{\prod_{m=1}^{3}(s_L-Sr_m)}{\prod_{l=1}^{4}(s_L-Sp_l)}
\end{equation}
The numerator has been scaled to make DC gain 1
\subsection{The analog bandstop filter transfer function}
The transformation equation:
\begin{equation}
    s_L= \frac{Bs}{s^2+\Omega_{0}^2} = \frac{0.8615*s}{s^2+1.2078}
\end{equation}
Thus, the transfer function:
\begin{equation}
    H_{analog,BSF}(s)=2.5492*\frac{\prod_{i=0}^{2}(s^2-\frac{Bs}{Sp_l}+\Omega_{0}^2)}{\prod_{i=0}^{2}(s^2-\frac{Bs}{Sp_l}+\Omega_{0}^2)} 
\end{equation}
\\Factorizing the denominator to get new roots so that the numerator and denominator can each be expressed as a product of 6 monomials.
\\We get the new poles and roots as:
\begin{equation}
    pol_i=\frac{\frac{B}{Sp_l}\pm \sqrt{(\frac{B}{Sp_l})^2-4\Omega_{0}^2}}{2};rot_i=\frac{\frac{B}{Sr_m}\pm \sqrt{(\frac{B}{Sr_m})^2-4\Omega_{0}^2}}{2}
\end{equation}

    

Thus the new analog filter transfer function where $pol_i$ are poles and $rot_i$ are roots of the analog bandstop filter is:
\begin{equation}
    H_{analog,BSF}(s)=2.5492*\frac{\prod_{i=0}^{5}(s-rot_i)}{\prod_{i=0}^{5}(s-pol_i)} 
\end{equation}

We get the new poles as: \textbf{\\-0.0676741640594894 + 1.60615952902148i
\\-0.0676741640594894 - 1.60615952902148i
\\-0.691241522865993 + 0.854392273528913i
\\-0.691241522865993 - 0.854392273528913i
\\-0.0316279187101118 + 0.750648105157324i
\\-0.0316279187101118 - 0.750648105157324i}
\\We get the new roots as: \textbf{\\0.00000000000000 + 1.49267032845577i
\\0.00000000000000 - 1.49267032845577i
\\9.71445146547012e-17 + 1.09900000000000i
\\9.71445146547012e-17 - 1.09900000000000i
\\-1.11022302462516e-16 + 0.809154558092894i
\\-1.11022302462516e-16 - 0.809154558092894i}
    
\subsection{The discrete time filter transfer function}
The bilinear transform from analog to discrete domain is:
\begin{equation}
    \frac{1-z^{-1}}{1+z^{-1}}
\end{equation}
On substituting this s in equation for the analog Bandstop filter, we get:
\begin{equation}
    H_{discrete,BPF}(s)=2.549*\frac{\prod_{i=0}^{5}((1-rot_i)-(1+rot_i)z^{-1})}{\prod_{i=0}^{5}((1-pol_i)-(1+pol_i)z^{-1})} 
\end{equation}
Thus we get new the poles and roots at:
\begin{equation}
    p_i=\frac{1+pol_i}{1-pol_i};  r_i=\frac{1+rot_i}{1-rot_i}
\end{equation}

 \\We get the new poles as: \textbf{\\0.267567369280753 + 0.922325798529715i
\\0.267567369280753 - 0.922325798529715i
\\-0.425931718110352 + 0.863601716988509i
\\-0.425931718110352 - 0.863601716988509i
\\-0.0578787074652288 + 0.475946895925734i
\\-0.0578787074652288 - 0.475946895925734i}
\\We get the new roots as: \textbf{\\0.208655594465270 + 0.977989183426046i
\\0.208655594465270 - 0.977989183426046i
\\-0.380433733517267 + 0.924808182490895i
\\-0.380433733517267 - 0.924808182490895i
\\-0.0941212545877101 + 0.995560741208108i
\\-0.0941212545877101 - 0.995560741208108i}

\\We have the values of all coefficients for equation (67) and thus we can substitute and find the values of the coefficients of the discrete bandstop filter transfer function.

\begin{center}
\includegraphics[scale=0.35]{qr.jpg}
\end{center}

\subsection{Code}
\begin{verbatim}
close all;clearvars;clc;

%Specifications
fp1 = 87;fs1 = 92;fs2 = 132;fp2 = 137;f_samp = 425;
B1=0.8615;W0=1.099;
Ws=1.2578;Wp=1;
Gp = 0.85;Gs = 0.15;

%Calculating N
ep = sqrt(1/Gp^2 - 1); es = sqrt(1/Gs^2 - 1);
k = Wp/Ws ;k1 = ep/es;
[K,Kp] = ellipk(k);[K1,K1p] = ellipk(k1);
Nexact = (K1p/K1)/(Kp/K);N = ceil(Nexact); 

%Elliptical approximation
Ap=1.4116;As=16.4782;
[z,p,H0,B,A] = ellipap2(N,Ap,As);
[y1,~] = size(B);
num = B(1,:);den = A(1,:);
if y1>1
    for i=2:y1
        num=conv(num,B(i,:));
        den=conv(den,A(i,:));
    end
end
syms s z;

%Analog LPF
analog_lpf(s) = poly2sym(num,s)/poly2sym(den,s);  
[ns, ds] = numden(analog_lpf(s));                                        
ns = sym2poly(expand(ns));
ds = sym2poly(expand(ds));                             
kd = ds(1);kn = ns(1);    
ds = ds/k;ns = ns/k;
fvtool(ns,ds,'Analysis','freq');                       
[zs,ps,]=tf2zp(ns,ds);

%Analog BSF
analog_bpf(s) = analog_lpf((s*s + W0*W0)/(B1*s));   
[nsb, dsb] = numden(analog_bpf(s));                                        
nsb = sym2poly(expand(nsb));
dsb = sym2poly(expand(dsb));                             
kd = dsb(1);kn = nsb(1);    
dsb = dsb/k;nsb = nsb/k;
fvtool(nsb,dsb,'Analysis','freq');                       
[zsb,psb,]=tf2zp(nsb,dsb);
k=kn/kd;

%Discrete BPF
discrete_bpf(z) = analog_bpf((z-1)/(z+1));              
[nz, dz] = numden(discrete_bpf(z));                                        
nz = sym2poly(expand(nz));
dz = sym2poly(expand(dz));                             
k = dz(1);    
dz = dz/k;nz = nz/k;
fvtool(nz,dz,'Analysis','freq');                       
[z,p,]=tf2zp(nz,dz);

%Pole zero plot
figure;
plot(real(p),imag(p),'rX');
hold on
plot(real(z),imag(z),'bO');
title("Poles and roots of the transfer function")
xlabel("Real")
ylabel("Imaginary")
axis equal
grid on
t = linspace(0,2*pi,1000);
hold on
plot(cos(t),sin(t),'b-') 


%Magnitude and phase plot
[H,f] = freqz(nz,dz,1024*1024, f_samp);
figure;
plot(f,abs(H),'LineWidth',1);
hold on;
title("Magnitude Response vs unnormalized frquency")
xlabel("1000 Hz")
ylabel("Response")
xline(fs1);xline(fp1);xline(fp2);xline(fs2);
yline(0.85);yline(0.15);
grid
\end{verbatim}
\subsection{Comparison of results with Chebyshev filter}
\subsubsection{Magnitude vs unnormalized frequency response }
\begin{center}
\includegraphics[scale=0.4]{qq.jpg}
\end{center}
\textbf{Corresponding Chebyshev graph}
\begin{center}
\includegraphics[scale=1]{ac (1).jpg}
\end{center}
\subsubsection{Magnitude and phase response vs normalized frequency}
\begin{center}
\includegraphics[scale=0.4]{qt.jpg}
\end{center}
\textbf{Corresponding Chebyshev graph}
\begin{center}
\includegraphics[scale=0.7]{ae (1).jpg}
\includegraphics[scale=0.7]{ad (1).jpg}
\end{center}
\subsubsection{Comparisons}
\begin{itemize}
    \item In both realizations, the specifications are satisfied and we can observe fs1,fs2,fp1 \& fp2.
    \item The Chebyshev filter has a equiripple passband and a monotonic stopband while the elliptical filter has equiripple passband and stopband.
    \item The Chebyshev filter had an order of 10 in both the numerator and denominator, thus
requiring around 18 delay stage For the elliptical filter, we have a power of 6 in both numerator and denominator, thus requiring only around 12 delays. Thus, the elliptical filter is more resource efficient than the chebyshev filter. (lower order)
\end{itemize}



\section{Peer Review}
\textbf{I have reviewed the report of Sameep Chattopadhyay (20d070067) and have found it to be correct.} The filter design steps were completed and the phase and magnitude response plots were present. He has started from the un-normalised filter, then normalized it, then converted to an analog filter, which was converted to a lowpass filter, to which he applied a frequency transform and then reconverted it to a bandpass  filter in the first case and a bandstop filter in the second case.

\end{document}
